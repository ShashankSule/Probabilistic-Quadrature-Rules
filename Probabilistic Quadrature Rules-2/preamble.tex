\usepackage[utf8]{inputenc}
\usepackage{amsmath}
\usepackage{amsthm}
\usepackage{amssymb}
\usepackage{amsfonts}
\usepackage{xcolor}
\usepackage{verbatim}
\usepackage{multicol}
\usepackage{graphicx}
\usepackage{booktabs}
\usepackage{float}
\usepackage{listings}
\lstset{language=Matlab,%
    %basicstyle=\color{red},
    breaklines=true,%
    morekeywords={matlab2tikz},
    keywordstyle=\color{blue},%
    morekeywords=[2]{1}, keywordstyle=[2]{\color{black}},
    identifierstyle=\color{black},%
    stringstyle=\color{mylilas},
    commentstyle=\color{mygreen},%
    showstringspaces=false,%without this there will be a symbol in the places where there is a space
    numbers=left,%
    numberstyle={\tiny \color{black}},% size of the numbers
    numbersep=9pt, % this defines how far the numbers are from the text
    emph=[1]{for,end,break},emphstyle=[1]\color{red}, %some words to emphasise
    %emph=[2]{word1,word2}, emphstyle=[2]{style},    
}
%\usepackage{times}
%\theoremstyle{plain}
%   \newtheorem{theorem}{Theorem}[section]
%   \newtheorem{proposition}[theorem]{Proposition}
%   \newtheorem{axiom}{Axiom}
%   \newtheorem{result}{Result}
%   \newtheorem{lemma}[theorem]{Lemma}
%   \newtheorem{corollary}[theorem]{Corollary}
%   \newtheorem{conjecture}[theorem]{Conjecture}
%   \newtheorem{problem}[theorem]{Problem}
%   \newtheorem{claim}[theorem]{Claim}
%\theoremstyle{definition}
%   \newtheorem{definition}{Definition}
%   \newtheorem{technique}{Technique}
%\newtheorem{example}{Example}
%\newenvironment{solution}
%  {\begin{proof}[Solution]}
%  {\end{proof}}
% \definecolor{green}{RGB}{34, 139, 34}
 
%\pagenumbering{gobble}
 
\theoremstyle{remark}
\newcommand*{\QEDB}{\hfill\ensuremath{\square}}
\newtheorem{remark}{Remark}[section]
\renewcommand{\ss}{\mathbf{s}}
\newcommand{\BB}{\mathcal{B}}
\newcommand{\xx}{\mathbf{x}}
\newcommand{\LL}{\mathcal{L}}
\newcommand{\II}{\mathcal{I}}
\newcommand{\MM}{\mathcal{M}}
\newcommand{\yy}{\mathbf{y}}
\newcommand{\zz}{\mathbf{z}}
\newcommand{\vv}{\mathbf{v}}
\newcommand{\ww}{\mathbf{w}}
\newcommand{\NN}{\mathbb{N}}
\newcommand{\FF}{\mathcal{F}}
\newcommand{\PP}{\mathcal{P}}
\newcommand{\QQQ}{\mathcal{Q}}
\newcommand{\GG}{\mathcal{G}}
\newcommand{\PO}{\mathbb{P}}
\newcommand{\eps}{\epsilon}
\newcommand{\HH}{\mathcal{H}}
\newcommand{\quo}{\bign /}
\newcommand{\EE}{\mathcal{E}}
\newcommand{\HHH}{\overline{\mathcal{H}}}
\newcommand{\ZZ}{\mathbb{Z}}
\newcommand{\QQ}{\mathbb{Q}}
\newcommand{\RR}{\mathbb{R}}
\newcommand{\CC}{\mathbb{C}}
\newcommand{\NNN}{\mathcal{N}}
\newcommand{\CM}{\mathbb{C}_{ {MA}}}
\newcommand{\CMS}{\mathbb{C}_{ {MAS}}}
\newcommand{\sym}{\mathfrak{S}}
\newcommand{\om}{\omega}
\newcommand{\vectorthree}[3]{\begin{bmatrix} #1\\#2\\#3 \end{bmatrix}}
\newcommand{\vectortwo}[2]{\begin{bmatrix} #1\\#2 \end{bmatrix}}
\let\Vec\mathbf
\newcommand{\Cross}[2]{\Vec{#1}\times\Vec{#2}}
\newcommand{\DPartial}[2]{\frac{\partial #1}{\partial #2}}
\newcommand{\Metric}[2]{||#1 - #2||}
\newcommand{\Proj}[2]{\text{Proj}_{\Vec{#1}}(\Vec{#2})}
\newcommand{\Grad}[1]{\nabla #1}
\newcommand{\DDot}[2]{\Vec{#1}\cdot\Vec{#2}}
\newcommand{\Curl}[1]{\text{curl}\,\textbf{#1}}
\newcommand{\Div}[1]{\text{Div}\,\textbf{#1}}
\newcommand{\ps}{\text{ps}_q^1}
\renewcommand{\mod}{\mathop{\rm \ mod}}
\renewcommand{\Im}{{\rm Im}}
\renewcommand{\bar}{\overline}
\newcommand{\re}{\textbf{Re }}
\newcommand{\im}{\textbf{Im }}